Jau vairākus gadu desmitus datorikas zinātne cenšas atrisināt algoritmu atpazīšanas problēmu. Tika pielietotas visai dažādas metodes un ar dažādiem rezultātiem. [TODO] Kāpēc mēs cenšamies panākt šo rezultātu? Kāds no tā būtu labums? 

Arvien biežāk tiek dzirdēts par komunikāciju starp datoriem un citām iekārtām bez cilvēka starpniecības. Šie datori apstrādā datus, iegūst rezultātus un padod tos nākošajam. Diemžēl, ļoti bieži šo mašīnu programmatūra ir novecojusi jau tajā brīdī, kad tā sāk strādāt. Pasaule nemitīgi attīstās, programmatūrai vajadzētu tai sekot un pielāgoties. Šeit ir problēma - no klasiskā skatpunkta - programma ir gatavs produkts, kad tā tiek nokompilēta un palaista, tā nemainīsies. Kādā veidā, tad tā varētu pielāgoties?

Kad programma spēs atpazīt savu saturu, to varētu arī "iemācīt" attiecīgi sevi mainīt un pielāgoties. Šādā veidā radot iespēju programmatūrai iet laikam līdzi, nevis novecot ar katru dienu, kad to sāka plānot. Būsim godīgi, tas risinājums, kurš tiek pieņemts kā pareizas šodien, ļoti sarežģītām un lielām sistēmām, kuras jāveido ilgāku laiku, būs novecojis uz to brīdi, kad programmatūra tiks pabeigta. 

Dotā risinājuma fokuss tiek virzīts uz kārtošanas algoritmiem, precīzāk - to atšķiršanu. Mērķis ir panākt to, lai radītā programmatūra varētu atpazīt ieejas datu saturu un pateikt kurš no visiem algoritmiem tas varētu būt. Tai brīdī, kad tiktu iegūts risinājums, kurš spēj atpazīt vienkāršus algoritmus, to varētu pielāgot sarežģītākiem. 

Darbs sastāv no ievada, X nodaļām un secinājumiem:
\begin{itemize}
    \item Nodaļā "{\@nodone}" tiek dziļāk izklāstīti darba mērķi un paredzamās problēmas.
    \item Nodaļā "{\@nodtwo}" tiek apskatīti citu autoru darbi saistībā ar līdzīgām problēmām un mērķiem. 
    \item Nodaļā "{\@nodthree}" tiek stāstīts par autora piedāvāto risinājumu, dažādu problēmu teorētiskiem risinājumiem.
    \item Nodaļā "{\@nodfour}" tiek izklāstīts tehniskais risinājums, tai skaitā problēmu teorētisko risinājumu realizācija.
    \item Nodaļā "{\@nodfive}" tiek apskatīti risinājuma potenciāli un pielietojumi.
    \item Nodaļā "{\@nodsix}" tiek uzskaitīti apkopoti rezultāti un izvērsti secinājumi.
\end{itemize} 